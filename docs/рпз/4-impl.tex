\chapter{Технологический раздел}
\label{cha:design}

\section{Выбор языка программирования и среды разработки}

В данной курсовой работе использовался язык программирования - c \cite{bib1}.

В качестве среды разработки я использовала Visual Studio Code \cite{bib2}, т.к. считаю его достаточно удобным и легким.
Visual Studio Code подходит не только для  Windows \cite{bib3}, но и для Linux \cite{bib4}, это еще одна причина, по которой я выбрала VS code,
т.к. у меня установлена ОС Ubuntu 18.04.4 \cite{bib5}.

\section{Листинг загружаемого модуля ядра}

На листинге 3.1 приведен код загружаемого моду ядра. 

\begin{lstlisting}[language=c, label=some-code, caption=Загружаемый модуль ядра]
MODULE_LICENSE("GPL");
MODULE_AUTHOR("Alice");
MODULE_DESCRIPTION("Coursework BMSTU");

#define KBD_DATA_REG 0x60 /* I/O port for keyboard data */
#define KBD_SCANCODE_MASK 0x7f
#define KBD_STATUS_MASK 0x80


#define ANALYSIS_PROGRAM_NAME "my_mem_prog"

#define MONITORING_SCANCODE 1 // "[ESC]"

unsigned int my_irq = 1; // Прерывание от клавиатуры.

static struct workqueue_struct *my_wq; //очередь работ
struct task_struct *task = NULL;
char current_scancode;

bool find_user_task_struct(char* prog_name);
static void my_wq_function(struct work_struct *work);
irqreturn_t irq_handler(int irq, void *dev, struct pt_regs *regs);
void update_info_about_mem(struct mm_struct *info_about_mem);

DECLARE_WORK(my_work, my_wq_function);

unsigned long  total_vm_old;
unsigned long  locked_vm_old;
int  map_count_old;
unsigned long  all_brk_old;

void info_about_mm(void)
{
	struct mm_struct *info_about_mem; 
	bool is_new_data = false;

	unsigned long  total_vm_current;
	unsigned long  locked_vm_current;
	int map_count_current;
	unsigned long  all_brk_current;

	if (find_user_task_struct(ANALYSIS_PROGRAM_NAME) == false)
	{
		printk(KERN_INFO "+Module: find_user_task_struct is false");
		return;
	}

	info_about_mem = task->mm;

	total_vm_current = info_about_mem->total_vm;
	locked_vm_current = info_about_mem->locked_vm;
	map_count_current = info_about_mem->map_count;
	all_brk_current = info_about_mem->brk - info_about_mem->start_brk;
	
	// update_info_about_mem(info_about_mem);

	if (total_vm_old > total_vm_current)
	{
		total_vm_old = total_vm_current;
		return;
	}

	if (all_brk_old > all_brk_current)
	{
		all_brk_old = all_brk_current;
		return;
	}


	if (total_vm_current != total_vm_old)
	{
		printk(KERN_INFO "+Module: Общее количество страниц памяти: было: %lu; стало:%lu; разница:%lu", total_vm_old, total_vm_current, total_vm_current - total_vm_old);
		total_vm_old = total_vm_current;
		is_new_data = true;
	}

	if (all_brk_current != all_brk_old)
	{
		printk(KERN_INFO "+Module: Используется сегментом кучи: было: %lu; стало:%lu; разница:%lu", all_brk_old, all_brk_current, all_brk_current - all_brk_old);
		all_brk_old = all_brk_current;
		is_new_data = true;
	}

	if (!is_new_data)
	{
		printk(KERN_INFO "+Module: Нет изменений");
	}
}

static void my_wq_function(struct work_struct *work) // вызываемая функция
{
	int scan_normal;

	if (!(current_scancode & KBD_STATUS_MASK))
	{
		scan_normal = current_scancode & KBD_SCANCODE_MASK;

		if (scan_normal == MONITORING_SCANCODE)
		{
			info_about_mm();
		}
	}
}

irqreturn_t irq_handler(int irq, void *dev, struct pt_regs *regs)
{
	if (irq == my_irq)
	{		
		// Получаем скан-код нажатой клавиши.
		current_scancode = inb(KBD_DATA_REG);
		queue_work(my_wq, &my_work);

		return IRQ_HANDLED; // прерывание обработано
	}
	else
		return IRQ_NONE; // прерывание не обработано
}

bool find_user_task_struct(char* prog_name)
{
	struct task_struct *current_task = &init_task;

	do {
		if (!strcmp(current_task->comm, prog_name))
		{
			task = current_task;
			return true;
		}
	} while ((current_task = next_task(current_task)) != &init_task);

	return false;
}

void update_info_about_mem(struct mm_struct *info_about_mem)
{
	atomic_t mm_users; 
	int counter;

	mm_users = info_about_mem->mm_users; /* Счетчик использования адресного пространства */
	counter = mm_users.counter;

	total_vm_old = info_about_mem->total_vm;
	locked_vm_old = info_about_mem->locked_vm;
	map_count_old = info_about_mem->map_count;
	all_brk_old = info_about_mem->brk - info_about_mem->start_brk;

	printk(KERN_INFO "+Module: Количество процессов, в которых используется данное адресное пространство: %d", counter);

	printk(KERN_INFO "+Module: Общее количество страниц памяти = %lu ", total_vm_old);
	printk(KERN_INFO "+Module: Количество заблокированных страниц памяти = %lu ", locked_vm_old);
	printk(KERN_INFO "+Module: Количество областей памяти: %d", map_count_old);

	printk(KERN_INFO "+Module: Используется сегментом кучи: %lu", all_brk_old);
	printk(KERN_INFO "+Module: Используется сегментом кода: %lu", info_about_mem->end_code - info_about_mem->start_code);
	printk(KERN_INFO "+Module: Используется сегментом данных: %lu", info_about_mem->end_data - info_about_mem->start_data);
}


void first_proc(void)
{
	struct mm_struct *info_about_mem; 


	if (find_user_task_struct(ANALYSIS_PROGRAM_NAME) == false)
	{
		printk(KERN_INFO "+Module: find_user_task_struct is false");
		return;
	}

	info_about_mem = task->mm;

	update_info_about_mem(info_about_mem);
}

static int __init md_init(void)
{
	// регистрация обработчика прерывания
	if (request_irq(
			my_irq,						/* номер irq */
			(irq_handler_t)irq_handler, /* наш обработчик */
			IRQF_SHARED,				/* линия может быть раздедена, IRQ
										(разрешено совместное использование)*/
			"my_irq_handler",			/* имя устройства (можно потом посмотреть в /proc/interrupts)*/
			(void *)(irq_handler)))		/* Последний параметр (идентификатор устройства) irq_handler нужен
										для того, чтобы можно отключить с помощью free_irq  */
	{
		printk(" + Error request_irq");
		return -1;
	}

	my_wq = create_workqueue("my_queue"); //создание очереди работ
	if (my_wq)
	{
		printk(KERN_INFO "Module: workqueue created!\n");
	}
	else
	{
		free_irq(my_irq, irq_handler); // Отключение обработчика прерывания.
		printk(KERN_INFO "Module: error create_workqueue()!\n");
		return -1;
	}

	first_proc();
	// find_user_task_struct(ANALYSIS_PROGRAM_NAME);

	printk(KERN_INFO "Module: module loaded!\n");
	return 0;
}

static void __exit md_exit(void)
{
	// Принудительно завершаем все работы в очереди.
	// Вызывающий блок блокируется до тех пор, пока операция не будет завершена.
	flush_workqueue(my_wq);
	destroy_workqueue(my_wq);

	// my_irq - номер прерывания.
	// irq_handler - идентификатор устройства.
	free_irq(my_irq, irq_handler); // Отключение обработчика прерывания.
	
	printk(KERN_INFO "Module: unloaded!\n");
}

module_init(md_init);
module_exit(md_exit);
\end{lstlisting}

\section{Анализируемая программа}

На листинге 3.2 приведен код анализируемой программы. 
% gcc pthread_example.c -pthread -o my_mem_prog

\begin{lstlisting}[language=c, label=some-code, caption=Анализируемая программа]
#include <stdio.h>
#include <stdlib.h>
#include <time.h>
#include <unistd.h>
#include <errno.h>  
#include <pthread.h>

#define SUCCESS 0

#define ERROR_CREATE_THREAD -11
#define ERROR_JOIN_THREAD   -12

#define VALUE_SIZE 64
#define PTHREAD_COUNT 3
#define SLEEP_TIME 3

#define GET_RAND_NUMBER(min, max) (rand() % (max - min + 1) + min)

typedef struct Node {
	int *value; // VALUE_SIZE
	struct Node *next; // 8 byte
} Node;

Node *create()
{
	Node *node = (Node*)malloc(sizeof(Node));
	node->value = (int*)malloc(VALUE_SIZE * sizeof(int)); // VALUE_SIZE * 4 (при 64 == 256)
	node->next = NULL;
	return node;
}

void add_to_end(Node *node)
{
	Node *new_node = create();
	node->next = new_node;
}

void output_data(int *data)
{
	for (int i = 0; i < VALUE_SIZE; i++)
	{
		printf("%d ", data[i]);
	}
	printf("\n");
}

void output(Node *node)
{
	while(node != NULL)
	{
		output_data(node->value);
		node = node->next;
	}
	printf("\n\n");
}

void generate_random_values(Node *node)
{
	for (int i = 0; i < VALUE_SIZE; i++)
	{
		node->value[i] = GET_RAND_NUMBER(0, 100);
	}
}

void update_list(Node *node)
{
	while(node != NULL)
	{	
		generate_random_values(node);
		node = node->next;
	}
}

int msleep(long msec)
{
	struct timespec ts;
	int res;

	if (msec < 0)
	{
		errno = EINVAL;
		return -1;
	}

	ts.tv_sec = msec / 1000;
	ts.tv_nsec = (msec % 1000) * 1000000;

	do {
		res = nanosleep(&ts, &ts);
	} while (res && errno == EINTR);

	return res;
}

void process(char* name)
{
	Node *first = create();
	Node *current_node = first;

	long long int i = 0;
	while (1)
	{
		update_list(first);

		add_to_end(current_node);
		
		current_node = current_node->next;
		
		i++;
		if (!(i % 16))
		{
			printf("1 kilobytes\n");
		}

		long long int byte = (256 + 8) * i;
		printf(" %s byte = %lld kilobyte = %lld ",name, byte, byte / 1024);
		printf("pages = %lld \n", byte / 1024 / 4);
		msleep(SLEEP_TIME);
	}
}
	
void* do_pthread(void *args) 
{
	process((char*)args);
	return SUCCESS;
}

int main() 
{
	printf("Start program");
	srand(time(NULL));
	setbuf(stdout, NULL);
	
	pthread_t threads[PTHREAD_COUNT];
	char* names[5] = {"1", "2", "3", "4", "5"};

	int status;
	int status_addr;

	for (int i = 0; i < PTHREAD_COUNT; i++)
	{
		status = pthread_create(&threads[i], NULL, do_pthread, names[i]);
		if (status != 0) {
			printf("main error: can't create thread, status = %d\n", status);
			exit(ERROR_CREATE_THREAD);
		}
	}

	do_pthread("Main pthread"); 
}
\end{lstlisting}


% \section{Сведения о модулях программы}

% Данная программа разбита на модули:

% \begin{itemize}
% 	\item main.py - Файл, содержащий точку входа в программу. В нем происходит общение с пользователем и вызов алгоритмов;
% \end{itemize}

% На листингах 3.1-3.6 представлены 

% % \begin{lstlisting}[label=some-code,caption=Главная функция main]
% % \end{lstlisting}

% \section{Тестирование}

% В данном разделе будет приведена таблица 
% % \ref{table:ref1}, в которой четко отражено тестирование программы. ы

\section{Вывод}

В данном разделе был выбран язык программирования и среда разработки.
А также представлены листинги.